\documentclass[a4paper,14pt]{report}
%
%--------------------   start of the 'preamble'
%
\usepackage{graphicx,amssymb,amstext,amsmath}
\usepackage{adobe_fonts}
\usepackage{report_title}
\usepackage[table,usenames,dvipsnames,svgnames]{xcolor}

%
%%    homebrew commands -- to save typing
\newcommand\etc{\textsl{etc}}
\newcommand\eg{\textsl{eg.}\ }
\newcommand\etal{\textsl{et al.}}
\newcommand\Quote[1]{\lq\textsl{#1}\rq}
\newcommand\fr[2]{{\textstyle\frac{#1}{#2}}}
\newcommand\miktex{\textsl{MikTeX}}
\newcommand\comp{\textsl{The Companion}}
\newcommand\nss{\textsl{Not so Short}}
%
%---------------------   end of the 'preamble'
%

\renewcommand{\abstractname}{\LARGE{Abstract}}

\begin{document}
%-----------------------------------------------------------
\Letsmaketitle
%-----------------------------------------------------------
\begin{abstract}\centering
\vspace{3mm}
为什么不同乐器听起来的声音不同?为什么人耳能够区分同一频率的不同乐器?为什么人耳能区分同时奏响的不同频率的声音?为什么从声音中\\能听到情感的表达?自然界的声音可以完全用简洁的\\数学表达吗?音乐数字化会遇到哪些问题?
\end{abstract}
%-----------------------------------------------------------
\tableofcontents
%-----------------------------------------------------------
\chapter{动机阐述}
\section{总体概述}
物理学引论(1)的课程中,在讲解振动的知识时涉及了一些声学的内容,以前在大家看来非常神奇的声现象都得到了初步的解释,尤其是许多学乐器或者爱听音乐的同学,在听了泛音这部分内容后,对音乐有了更深刻的理解。
\par
在本次研究中,我们从频谱图出发,对我们感兴趣的一些问题,如有没有客观存在的音高,人耳是如何理解声音的,尤其是如何同时听出不同音色和不同音高,频谱如何体现情感等问题进行研究,并得出一些初步的结论和进一步的猜想。
\section{研究动机}
这次大作业之所以选择这个题目,是因为我们认为作为非物理专业的学生,学习物理的主要目的是学习物理的思想,并对其他方面的学习产生指导作用,另一方面,也通过物理更深刻的了解这个世界,更深入的理解生活。
\par
因此,我们选取了我们认为最贴近生活的主题——音乐,来进行研究。在进行一定量的科学分析之后,我们认为更主要的是深入的思考,是更贴近于对哲学与生活的思考,以及如何解释身边的现象,如何认识科学与感知的关系等。

\chapter{研究过程}
\section{对不同乐器频谱图的分析}


\chapter{成果概述}
\section{总结}

\begin{itemize}
    \item 通过分析五种乐器的频谱图,巩固了对泛音列的学习, 并对泛音列的一些特性进行了研究
    \item 通过观察大提琴的波形图,认识到同一乐器不同音高的音波形也可以有很大差别,音的起始部分变化剧烈十分复杂,延音部分有很强的规律可循
    \item 通过对多个波形图与实际的对比,得出决定客观听觉音高的是波形的最小正周期,并研究了“基音消失”的现象,再次确认了最小正周期才是决定听觉音高的要素
    \item 通过分析不同频率波形的合成图,对人耳辨别同时奏响的不同频率的声音的能力的原理进行了猜想,并对辨别不同乐器的能力也进行了解释。
    \item 通过查阅资料,知道了早期的音乐数字化通过FM实现,后来通过波表实现,并对波表的要求进行了分析
    \item 对音乐与听感的关系进行了思考与解释
\end{itemize}

\section{感谢}
\begin{itemize}
    \item 感谢李晟老师一学期的辛勤教学,让我们受益良多。尤其是关于物理与哲学的思辨给了我们许多指引
    \item 感谢组员们的相互配合,共同完成这次大作业
\end{itemize}

%-----------------------------------------------------------
\appendix
\chapter{Testing Scripts}
该脚本是用bash编写的,每次读取当前要测试的主程序名(如MainArrayList),初始数据规模(如10000),最大数据规模(如100000000),脚本对于每个数据规模分别运行Cpp与Java的主程序,输出运行时间。
\par
\vspace{4mm}
\noindent
\mbox{}\textit{\textcolor{Brown}{\#!/bin/bash}} \\
\mbox{} \\
\mbox{}echo\ -n\ \texttt{\textcolor{Red}{"{}Your\ program\ name:\ "{}}} \\
\mbox{}\textbf{\textcolor{Blue}{read}}\ program$\_$name \\
\mbox{}echo\ -n\ \texttt{\textcolor{Red}{"{}Begin\ Value:\ "{}}} \\
\mbox{}\textbf{\textcolor{Blue}{read}}\ beginValue \\
\mbox{}echo\ -n\ \texttt{\textcolor{Red}{"{}End\ Value:\ "{}}} \\
\mbox{}\textbf{\textcolor{Blue}{read}}\ endValue \\
\mbox{} \\
\mbox{}echo\ \texttt{\textcolor{Red}{"{}"{}}} \\
\mbox{}echo\ \texttt{\textcolor{Red}{"{}\#\#\#\#\#\#\#\#\#\#\#\#\#\#\#\#\#\#\#\#\#\#\#\#\#\#\#\#\#\#\#\#"{}}} \\
\mbox{}echo\ \texttt{\textcolor{Red}{"{}\ \ \ \ \ \ \ \ \ Testing\ starts...\ \ \ \ \ \ "{}}} \\
\mbox{}echo\ \texttt{\textcolor{Red}{"{}"{}}} \\
\mbox{} \\
\mbox{}\textcolor{ForestGreen}{num}\textcolor{BrickRed}{=}\textcolor{Purple}{0}\textcolor{BrickRed}{;} \\
\mbox{}\textbf{\textcolor{Blue}{for}}\ \textcolor{BrickRed}{((}\textcolor{ForestGreen}{N}\textcolor{BrickRed}{=}\textcolor{ForestGreen}{\$beginValue}\textcolor{BrickRed}{;}\ N\textcolor{BrickRed}{\textless{}=}\textcolor{ForestGreen}{\$endValue}\textcolor{BrickRed}{;}\ \textcolor{ForestGreen}{N}\textcolor{BrickRed}{=}N\textcolor{BrickRed}{*}\textcolor{Purple}{10}\textcolor{BrickRed}{))} \\
\mbox{}\textbf{\textcolor{Blue}{do}} \\
\mbox{}\ \ \ \ \ \ \ \ \textcolor{BrickRed}{((}num\textcolor{BrickRed}{++))} \\
\mbox{}\ \ \ \ \ \ \ \ echo\ \texttt{\textcolor{Red}{"{}TEST\ CASE\ \$num"{}}} \\
\mbox{}\ \ \ \ \ \ \ \ echo\ \texttt{\textcolor{Red}{"{}size:\ "{}}}\textcolor{ForestGreen}{\$N} \\
\mbox{}\ \ \ \ \ \ \ \ echo\ -n\ \texttt{\textcolor{Red}{"{}My\ Cpp\ STL:\ "{}}} \\
\mbox{}\ \ \ \ \ \ \ \ \textcolor{BrickRed}{(}time\ -p\ \textcolor{BrickRed}{./}\textcolor{ForestGreen}{\$program$\_$name}\ \textcolor{ForestGreen}{\$N}\textcolor{BrickRed}{)}\ \textcolor{Purple}{2}\textcolor{BrickRed}{\textgreater{}\&}\textcolor{Purple}{1}\ \textcolor{BrickRed}{$|$}\ grep\ real\ \textcolor{BrickRed}{$|$}\ sed\ \texttt{\textcolor{Red}{'s/real/TIME/'}} \\
\mbox{}\ \ \ \ \ \ \ \ echo\ -n\ \texttt{\textcolor{Red}{"{}Offical\ Java\ STL:\ "{}}} \\
\mbox{}\ \ \ \ \ \ \ \ \textcolor{BrickRed}{(}time\ -p\ java\ \textcolor{ForestGreen}{\$program$\_$name}\ \textcolor{ForestGreen}{\$N}\textcolor{BrickRed}{)}\ \textcolor{Purple}{2}\textcolor{BrickRed}{\textgreater{}\&}\textcolor{Purple}{1}\ \textcolor{BrickRed}{$|$}\ grep\ real\ \textcolor{BrickRed}{$|$}\ sed\ \texttt{\textcolor{Red}{'s/real/TIME/'}} \\
\mbox{}\ \ \ \ \ \ \ \ echo\ \texttt{\textcolor{Red}{"{}"{}}} \\
\mbox{}\textbf{\textcolor{Blue}{done}} \\
\mbox{} \\
\mbox{}echo\ \texttt{\textcolor{Red}{"{}\ \ \ \ \ \ \ \ \ All\ tests\ done.\ \ \ \ \ \ "{}}} \\
\mbox{}echo\ \texttt{\textcolor{Red}{"{}\#\#\#\#\#\#\#\#\#\#\#\#\#\#\#\#\#\#\#\#\#\#\#\#\#\#\#\#\#\#\#\#"{}}} \\
\mbox{}echo\ \texttt{\textcolor{Red}{"{}"{}}} \\
\mbox{}

\chapter{Testing Main Programs}

每个Java与Cpp的主程序都从main参数读入数据规模N,然后运行。不执行IO。

\section{MainArrayList}
\subsection{MainArrayList.cpp}
\noindent
\mbox{}\textcolor{ForestGreen}{int}\ \textbf{\textcolor{Black}{main}}\textcolor{BrickRed}{(}\textcolor{ForestGreen}{int}\ argc\textcolor{BrickRed}{,}\ \textcolor{ForestGreen}{char}\ \textcolor{BrickRed}{**}argv\textcolor{BrickRed}{)}\ \textcolor{Red}{\{} \\
\mbox{} \\
\mbox{}\ \ \ \ \textcolor{ForestGreen}{int}\ N\ \textcolor{BrickRed}{=}\ \textbf{\textcolor{Black}{atoi}}\textcolor{BrickRed}{(}argv\textcolor{BrickRed}{[}\textcolor{Purple}{1}\textcolor{BrickRed}{]);} \\
\mbox{}\ \ \ \ \textcolor{ForestGreen}{int}\ tot\ \textcolor{BrickRed}{=}\ N\ \textcolor{BrickRed}{/}\ \textcolor{Purple}{10}\textcolor{BrickRed}{;} \\
\mbox{} \\
\mbox{}\ \ \ \ \textcolor{TealBlue}{ArrayList\textless{}int\textgreater{}}\ arr\textcolor{BrickRed}{;} \\
\mbox{}\ \ \ \ \textbf{\textcolor{Blue}{for}}\ \textcolor{BrickRed}{(}\textcolor{ForestGreen}{int}\ te\ \textcolor{BrickRed}{=}\ \textcolor{Purple}{0}\textcolor{BrickRed}{;}\ te\ \textcolor{BrickRed}{\textless{}}\ \textcolor{Purple}{10}\textcolor{BrickRed}{;}\ \textcolor{BrickRed}{++}te\textcolor{BrickRed}{)}\ \textcolor{Red}{\{} \\
\mbox{}\ \ \ \ \ \ \ \ \textbf{\textcolor{Blue}{for}}\ \textcolor{BrickRed}{(}\textcolor{ForestGreen}{int}\ i\ \textcolor{BrickRed}{=}\ \textcolor{Purple}{1}\textcolor{BrickRed}{;}\ i\ \textcolor{BrickRed}{\textless{}=}\ tot\textcolor{BrickRed}{;}\ \textcolor{BrickRed}{++}i\textcolor{BrickRed}{)}\ arr\textcolor{BrickRed}{.}\textbf{\textcolor{Black}{add}}\textcolor{BrickRed}{(}i\textcolor{BrickRed}{);} \\
\mbox{}\ \ \ \ \ \ \ \ \textbf{\textcolor{Blue}{for}}\ \textcolor{BrickRed}{(}\textcolor{ForestGreen}{int}\ i\ \textcolor{BrickRed}{=}\ \textcolor{Purple}{0}\textcolor{BrickRed}{;}\ i\ \textcolor{BrickRed}{\textless{}}\ arr\textcolor{BrickRed}{.}\textbf{\textcolor{Black}{size}}\textcolor{BrickRed}{();}\ \textcolor{BrickRed}{++}i\textcolor{BrickRed}{)}\ arr\textcolor{BrickRed}{.}\textbf{\textcolor{Black}{get}}\textcolor{BrickRed}{(}i\textcolor{BrickRed}{);} \\
\mbox{}\ \ \ \ \ \ \ \ \textbf{\textcolor{Blue}{while}}\ \textcolor{BrickRed}{(!}arr\textcolor{BrickRed}{.}\textbf{\textcolor{Black}{isEmpty}}\textcolor{BrickRed}{())}\ arr\textcolor{BrickRed}{.}\textbf{\textcolor{Black}{removeIndex}}\textcolor{BrickRed}{(}arr\textcolor{BrickRed}{.}\textbf{\textcolor{Black}{size}}\textcolor{BrickRed}{()}\ \textcolor{BrickRed}{-}\ \textcolor{Purple}{1}\textcolor{BrickRed}{);} \\
\mbox{}\ \ \ \ \textcolor{Red}{\}} \\
\mbox{} \\
\mbox{}\ \ \ \ \textbf{\textcolor{Blue}{return}}\ \textcolor{Purple}{0}\textcolor{BrickRed}{;} \\
\mbox{} \\
\mbox{}\textcolor{Red}{\}} \\
\mbox{} \\
\subsection{MainArrayList.java}
\noindent
\mbox{}\textbf{\textcolor{Blue}{public}}\ \textbf{\textcolor{Blue}{class}}\ \textcolor{TealBlue}{MainArrayList}\ \textcolor{Red}{\{} \\
\mbox{}\ \ \ \ \textbf{\textcolor{Blue}{public}}\ \textbf{\textcolor{Blue}{static}}\ \textcolor{ForestGreen}{void}\ \textbf{\textcolor{Black}{main}}\textcolor{BrickRed}{(}\textcolor{TealBlue}{String}\ args\textcolor{BrickRed}{[])}\ \textcolor{Red}{\{} \\
\mbox{}\ \ \ \ \ \ \ \  \\
\mbox{}\ \ \ \ \ \ \ \ \textcolor{ForestGreen}{int}\ N\ \textcolor{BrickRed}{=}\ Integer\textcolor{BrickRed}{.}\textbf{\textcolor{Black}{parseInt}}\textcolor{BrickRed}{(}args\textcolor{BrickRed}{[}\textcolor{Purple}{0}\textcolor{BrickRed}{]);} \\
\mbox{}\ \ \ \ \ \ \ \ \textcolor{ForestGreen}{int}\ tot\ \textcolor{BrickRed}{=}\ N\ \textcolor{BrickRed}{/}\ \textcolor{Purple}{10}\textcolor{BrickRed}{;} \\
\mbox{} \\
\mbox{}\ \ \ \ \ \ \ \ \textcolor{TealBlue}{ArrayList\textless{}Integer\textgreater{}}\ arr\ \textcolor{BrickRed}{=}\ \textbf{\textcolor{Blue}{new}}\ ArrayList\textcolor{BrickRed}{\textless{}}Integer\textcolor{BrickRed}{\textgreater{}();} \\
\mbox{}\ \ \ \ \ \ \ \ \textbf{\textcolor{Blue}{for}}\ \textcolor{BrickRed}{(}\textcolor{ForestGreen}{int}\ te\ \textcolor{BrickRed}{=}\ \textcolor{Purple}{0}\textcolor{BrickRed}{;}\ te\ \textcolor{BrickRed}{\textless{}}\ \textcolor{Purple}{10}\textcolor{BrickRed}{;}\ \textcolor{BrickRed}{++}te\textcolor{BrickRed}{)}\ \textcolor{Red}{\{} \\
\mbox{}\ \ \ \ \ \ \ \ \ \ \ \ \textbf{\textcolor{Blue}{for}}\ \textcolor{BrickRed}{(}\textcolor{ForestGreen}{int}\ i\ \textcolor{BrickRed}{=}\ \textcolor{Purple}{1}\textcolor{BrickRed}{;}\ i\ \textcolor{BrickRed}{\textless{}=}\ tot\textcolor{BrickRed}{;}\ \textcolor{BrickRed}{++}i\textcolor{BrickRed}{)}\ arr\textcolor{BrickRed}{.}\textbf{\textcolor{Black}{add}}\textcolor{BrickRed}{(}i\textcolor{BrickRed}{);} \\
\mbox{}\ \ \ \ \ \ \ \ \ \ \ \ \textbf{\textcolor{Blue}{for}}\ \textcolor{BrickRed}{(}\textcolor{ForestGreen}{int}\ i\ \textcolor{BrickRed}{=}\ \textcolor{Purple}{0}\textcolor{BrickRed}{;}\ i\ \textcolor{BrickRed}{\textless{}}\ arr\textcolor{BrickRed}{.}\textbf{\textcolor{Black}{size}}\textcolor{BrickRed}{();}\ \textcolor{BrickRed}{++}i\textcolor{BrickRed}{)}\ arr\textcolor{BrickRed}{.}\textbf{\textcolor{Black}{get}}\textcolor{BrickRed}{(}i\textcolor{BrickRed}{);} \\
\mbox{}\ \ \ \ \ \ \ \ \ \ \ \ \textbf{\textcolor{Blue}{while}}\ \textcolor{BrickRed}{(!}arr\textcolor{BrickRed}{.}\textbf{\textcolor{Black}{isEmpty}}\textcolor{BrickRed}{())}\ arr\textcolor{BrickRed}{.}\textbf{\textcolor{Black}{remove}}\textcolor{BrickRed}{(}arr\textcolor{BrickRed}{.}\textbf{\textcolor{Black}{size}}\textcolor{BrickRed}{()}\ \textcolor{BrickRed}{-}\ \textcolor{Purple}{1}\textcolor{BrickRed}{);} \\
\mbox{}\ \ \ \ \ \ \ \ \textcolor{Red}{\}} \\
\mbox{} \\
\mbox{}\ \ \ \ \textcolor{Red}{\}} \\
\mbox{}\textcolor{Red}{\}} \\
\mbox{}

\section{LinkedList}
\subsection{LinkedList.cpp}
\noindent
\mbox{}\textcolor{ForestGreen}{int}\ \textbf{\textcolor{Black}{main}}\textcolor{BrickRed}{(}\textcolor{ForestGreen}{int}\ argc\textcolor{BrickRed}{,}\ \textcolor{ForestGreen}{char}\ \textcolor{BrickRed}{**}argv\textcolor{BrickRed}{)}\ \textcolor{Red}{\{} \\
\mbox{} \\
\mbox{}\ \ \ \ \textcolor{ForestGreen}{int}\ N\ \textcolor{BrickRed}{=}\ \textbf{\textcolor{Black}{atoi}}\textcolor{BrickRed}{(}argv\textcolor{BrickRed}{[}\textcolor{Purple}{1}\textcolor{BrickRed}{]);} \\
\mbox{}\ \ \ \ \textcolor{ForestGreen}{int}\ tot\ \textcolor{BrickRed}{=}\ N\ \textcolor{BrickRed}{/}\ \textcolor{Purple}{10}\textcolor{BrickRed}{;} \\
\mbox{} \\
\mbox{}\ \ \ \ \textcolor{TealBlue}{LinkedList\textless{}int\textgreater{}}\ lnk\textcolor{BrickRed}{;} \\
\mbox{}\ \ \ \ \textbf{\textcolor{Blue}{for}}\ \textcolor{BrickRed}{(}\textcolor{ForestGreen}{int}\ te\ \textcolor{BrickRed}{=}\ \textcolor{Purple}{0}\textcolor{BrickRed}{;}\ te\ \textcolor{BrickRed}{\textless{}}\ \textcolor{Purple}{10}\textcolor{BrickRed}{;}\ \textcolor{BrickRed}{++}te\textcolor{BrickRed}{)}\ \textcolor{Red}{\{} \\
\mbox{}\ \ \ \ \ \ \ \ \textbf{\textcolor{Blue}{for}}\ \textcolor{BrickRed}{(}\textcolor{ForestGreen}{int}\ i\ \textcolor{BrickRed}{=}\ \textcolor{Purple}{1}\textcolor{BrickRed}{;}\ i\ \textcolor{BrickRed}{\textless{}=}\ tot\textcolor{BrickRed}{;}\ \textcolor{BrickRed}{++}i\textcolor{BrickRed}{)}\ lnk\textcolor{BrickRed}{.}\textbf{\textcolor{Black}{add}}\textcolor{BrickRed}{(}i\textcolor{BrickRed}{);} \\
\mbox{}\ \ \ \ \ \ \ \ \textbf{\textcolor{Blue}{while}}\ \textcolor{BrickRed}{(!}lnk\textcolor{BrickRed}{.}\textbf{\textcolor{Black}{isEmpty}}\textcolor{BrickRed}{())}\ lnk\textcolor{BrickRed}{.}\textbf{\textcolor{Black}{removeLast}}\textcolor{BrickRed}{();} \\
\mbox{}\ \ \ \ \textcolor{Red}{\}} \\
\mbox{} \\
\mbox{}\ \ \ \ \textbf{\textcolor{Blue}{return}}\ \textcolor{Purple}{0}\textcolor{BrickRed}{;} \\
\mbox{} \\
\mbox{}\textcolor{Red}{\}} \\
\mbox{}

\subsection{LinkedList.java}
\noindent
\mbox{}\textbf{\textcolor{Blue}{public}}\ \textbf{\textcolor{Blue}{class}}\ \textcolor{TealBlue}{MainLinkedList}\ \textcolor{Red}{\{} \\
\mbox{}\ \ \ \ \textbf{\textcolor{Blue}{public}}\ \textbf{\textcolor{Blue}{static}}\ \textcolor{ForestGreen}{void}\ \textbf{\textcolor{Black}{main}}\textcolor{BrickRed}{(}\textcolor{TealBlue}{String}\ args\textcolor{BrickRed}{[])}\ \textcolor{Red}{\{} \\
\mbox{}\ \ \ \ \ \ \ \  \\
\mbox{}\ \ \ \ \ \ \ \ \textcolor{ForestGreen}{int}\ N\ \textcolor{BrickRed}{=}\ Integer\textcolor{BrickRed}{.}\textbf{\textcolor{Black}{parseInt}}\textcolor{BrickRed}{(}args\textcolor{BrickRed}{[}\textcolor{Purple}{0}\textcolor{BrickRed}{]);} \\
\mbox{}\ \ \ \ \ \ \ \ \textcolor{ForestGreen}{int}\ tot\ \textcolor{BrickRed}{=}\ N\ \textcolor{BrickRed}{/}\ \textcolor{Purple}{10}\textcolor{BrickRed}{;} \\
\mbox{} \\
\mbox{}\ \ \ \ \ \ \ \ \textcolor{TealBlue}{LinkedList\textless{}Integer\textgreater{}}\ lnk\ \textcolor{BrickRed}{=}\ \textbf{\textcolor{Blue}{new}}\ LinkedList\textcolor{BrickRed}{\textless{}}Integer\textcolor{BrickRed}{\textgreater{}();} \\
\mbox{}\ \ \ \ \ \ \ \ \textbf{\textcolor{Blue}{for}}\ \textcolor{BrickRed}{(}\textcolor{ForestGreen}{int}\ te\ \textcolor{BrickRed}{=}\ \textcolor{Purple}{0}\textcolor{BrickRed}{;}\ te\ \textcolor{BrickRed}{\textless{}}\ \textcolor{Purple}{10}\textcolor{BrickRed}{;}\ \textcolor{BrickRed}{++}te\textcolor{BrickRed}{)}\ \textcolor{Red}{\{} \\
\mbox{}\ \ \ \ \ \ \ \ \ \ \ \ \textbf{\textcolor{Blue}{for}}\ \textcolor{BrickRed}{(}\textcolor{ForestGreen}{int}\ i\ \textcolor{BrickRed}{=}\ \textcolor{Purple}{1}\textcolor{BrickRed}{;}\ i\ \textcolor{BrickRed}{\textless{}=}\ tot\textcolor{BrickRed}{;}\ \textcolor{BrickRed}{++}i\textcolor{BrickRed}{)}\ lnk\textcolor{BrickRed}{.}\textbf{\textcolor{Black}{add}}\textcolor{BrickRed}{(}i\textcolor{BrickRed}{);} \\
\mbox{}\ \ \ \ \ \ \ \ \ \ \ \ \textbf{\textcolor{Blue}{while}}\ \textcolor{BrickRed}{(!}lnk\textcolor{BrickRed}{.}\textbf{\textcolor{Black}{isEmpty}}\textcolor{BrickRed}{())}\ lnk\textcolor{BrickRed}{.}\textbf{\textcolor{Black}{removeLast}}\textcolor{BrickRed}{();} \\
\mbox{}\ \ \ \ \ \ \ \ \textcolor{Red}{\}} \\
\mbox{}\ \ \ \ \textcolor{Red}{\}} \\
\mbox{}\textcolor{Red}{\}} \\
\mbox{}

\section{HashSet}
\subsection{HashSet.cpp}
\noindent
\mbox{}\textcolor{ForestGreen}{int}\ \textbf{\textcolor{Black}{main}}\textcolor{BrickRed}{(}\textcolor{ForestGreen}{int}\ argc\textcolor{BrickRed}{,}\ \textcolor{ForestGreen}{char}\ \textcolor{BrickRed}{**}argv\textcolor{BrickRed}{)}\ \textcolor{Red}{\{} \\
\mbox{} \\
\mbox{}\ \ \ \ \textcolor{ForestGreen}{int}\ N\ \textcolor{BrickRed}{=}\ \textbf{\textcolor{Black}{atoi}}\textcolor{BrickRed}{(}argv\textcolor{BrickRed}{[}\textcolor{Purple}{1}\textcolor{BrickRed}{]);} \\
\mbox{}\ \ \ \ \textcolor{ForestGreen}{int}\ tot\ \textcolor{BrickRed}{=}\ N\ \textcolor{BrickRed}{/}\ \textcolor{Purple}{10}\textcolor{BrickRed}{;} \\
\mbox{} \\
\mbox{}\ \ \ \ \textcolor{TealBlue}{HashSet\textless{}int,\ Hashint\textgreater{}}\ set\textcolor{BrickRed}{;} \\
\mbox{}\ \ \ \ \textbf{\textcolor{Blue}{for}}\ \textcolor{BrickRed}{(}\textcolor{ForestGreen}{int}\ te\ \textcolor{BrickRed}{=}\ \textcolor{Purple}{0}\textcolor{BrickRed}{;}\ te\ \textcolor{BrickRed}{\textless{}}\ \textcolor{Purple}{10}\textcolor{BrickRed}{;}\ \textcolor{BrickRed}{++}te\textcolor{BrickRed}{)}\ \textcolor{Red}{\{} \\
\mbox{}\ \ \ \ \ \ \ \ \textbf{\textcolor{Blue}{for}}\ \textcolor{BrickRed}{(}\textcolor{ForestGreen}{int}\ i\ \textcolor{BrickRed}{=}\ \textcolor{Purple}{1}\textcolor{BrickRed}{;}\ i\ \textcolor{BrickRed}{\textless{}=}\ tot\textcolor{BrickRed}{;}\ \textcolor{BrickRed}{++}i\textcolor{BrickRed}{)}\ set\textcolor{BrickRed}{.}\textbf{\textcolor{Black}{add}}\textcolor{BrickRed}{(}i\textcolor{BrickRed}{);} \\
\mbox{}\ \ \ \ \ \ \ \ \textbf{\textcolor{Blue}{for}}\ \textcolor{BrickRed}{(}\textcolor{ForestGreen}{int}\ i\ \textcolor{BrickRed}{=}\ \textcolor{Purple}{1}\textcolor{BrickRed}{;}\ i\ \textcolor{BrickRed}{\textless{}=}\ tot\textcolor{BrickRed}{;}\ \textcolor{BrickRed}{++}i\textcolor{BrickRed}{)}\ set\textcolor{BrickRed}{.}\textbf{\textcolor{Black}{remove}}\textcolor{BrickRed}{(}i\textcolor{BrickRed}{);} \\
\mbox{}\ \ \ \ \textcolor{Red}{\}} \\
\mbox{} \\
\mbox{}\ \ \ \ \textbf{\textcolor{Blue}{return}}\ \textcolor{Purple}{0}\textcolor{BrickRed}{;} \\
\mbox{} \\
\mbox{}\textcolor{Red}{\}} \\
\mbox{}
\subsection{HashSet.java}
\noindent
\mbox{}\textbf{\textcolor{Blue}{public}}\ \textbf{\textcolor{Blue}{class}}\ \textcolor{TealBlue}{MainHashSet}\ \textcolor{Red}{\{} \\
\mbox{}\ \ \ \ \textbf{\textcolor{Blue}{public}}\ \textbf{\textcolor{Blue}{static}}\ \textcolor{ForestGreen}{void}\ \textbf{\textcolor{Black}{main}}\textcolor{BrickRed}{(}\textcolor{TealBlue}{String}\ args\textcolor{BrickRed}{[])}\ \textcolor{Red}{\{} \\
\mbox{}\ \ \ \ \ \ \ \  \\
\mbox{}\ \ \ \ \ \ \ \ \textcolor{ForestGreen}{int}\ N\ \textcolor{BrickRed}{=}\ Integer\textcolor{BrickRed}{.}\textbf{\textcolor{Black}{parseInt}}\textcolor{BrickRed}{(}args\textcolor{BrickRed}{[}\textcolor{Purple}{0}\textcolor{BrickRed}{]);} \\
\mbox{}\ \ \ \ \ \ \ \ \textcolor{ForestGreen}{int}\ tot\ \textcolor{BrickRed}{=}\ N\ \textcolor{BrickRed}{/}\ \textcolor{Purple}{10}\textcolor{BrickRed}{;} \\
\mbox{} \\
\mbox{}\ \ \ \ \ \ \ \ \textcolor{TealBlue}{HashSet\textless{}Integer\textgreater{}}\ set\ \textcolor{BrickRed}{=}\ \textbf{\textcolor{Blue}{new}}\ HashSet\textcolor{BrickRed}{\textless{}}Integer\textcolor{BrickRed}{\textgreater{}();} \\
\mbox{}\ \ \ \ \ \ \ \ \textbf{\textcolor{Blue}{for}}\ \textcolor{BrickRed}{(}\textcolor{ForestGreen}{int}\ te\ \textcolor{BrickRed}{=}\ \textcolor{Purple}{0}\textcolor{BrickRed}{;}\ te\ \textcolor{BrickRed}{\textless{}}\ \textcolor{Purple}{10}\textcolor{BrickRed}{;}\ \textcolor{BrickRed}{++}te\textcolor{BrickRed}{)}\ \textcolor{Red}{\{} \\
\mbox{}\ \ \ \ \ \ \ \ \ \ \ \ \textbf{\textcolor{Blue}{for}}\ \textcolor{BrickRed}{(}\textcolor{ForestGreen}{int}\ i\ \textcolor{BrickRed}{=}\ \textcolor{Purple}{1}\textcolor{BrickRed}{;}\ i\ \textcolor{BrickRed}{\textless{}=}\ tot\textcolor{BrickRed}{;}\ \textcolor{BrickRed}{++}i\textcolor{BrickRed}{)}\ set\textcolor{BrickRed}{.}\textbf{\textcolor{Black}{add}}\textcolor{BrickRed}{(}i\textcolor{BrickRed}{);} \\
\mbox{}\ \ \ \ \ \ \ \ \ \ \ \ \textbf{\textcolor{Blue}{for}}\ \textcolor{BrickRed}{(}\textcolor{ForestGreen}{int}\ i\ \textcolor{BrickRed}{=}\ \textcolor{Purple}{1}\textcolor{BrickRed}{;}\ i\ \textcolor{BrickRed}{\textless{}=}\ tot\textcolor{BrickRed}{;}\ \textcolor{BrickRed}{++}i\textcolor{BrickRed}{)}\ set\textcolor{BrickRed}{.}\textbf{\textcolor{Black}{remove}}\textcolor{BrickRed}{(}i\textcolor{BrickRed}{);} \\
\mbox{}\ \ \ \ \ \ \ \ \textcolor{Red}{\}} \\
\mbox{} \\
\mbox{}\ \ \ \ \textcolor{Red}{\}} \\
\mbox{}\textcolor{Red}{\}} \\
\mbox{}

\section{TreeSet}
\subsection{TreeSet.cpp}
\noindent
\mbox{}\textcolor{ForestGreen}{int}\ \textbf{\textcolor{Black}{main}}\textcolor{BrickRed}{(}\textcolor{ForestGreen}{int}\ argc\textcolor{BrickRed}{,}\ \textcolor{ForestGreen}{char}\ \textcolor{BrickRed}{**}argv\textcolor{BrickRed}{)}\ \textcolor{Red}{\{} \\
\mbox{} \\
\mbox{}\ \ \ \ \textcolor{ForestGreen}{int}\ N\ \textcolor{BrickRed}{=}\ \textbf{\textcolor{Black}{atoi}}\textcolor{BrickRed}{(}argv\textcolor{BrickRed}{[}\textcolor{Purple}{1}\textcolor{BrickRed}{]);} \\
\mbox{}\ \ \ \ \textcolor{ForestGreen}{int}\ tot\ \textcolor{BrickRed}{=}\ N\ \textcolor{BrickRed}{/}\ \textcolor{Purple}{10}\textcolor{BrickRed}{;} \\
\mbox{} \\
\mbox{}\ \ \ \ \textbf{\textcolor{Black}{srand}}\textcolor{BrickRed}{(}\textbf{\textcolor{Black}{time}}\textcolor{BrickRed}{(}NULL\textcolor{BrickRed}{));} \\
\mbox{}\ \ \ \ \textcolor{TealBlue}{TreeSet\textless{}int\textgreater{}}\ tree\textcolor{BrickRed}{;} \\
\mbox{}\ \ \ \ \textbf{\textcolor{Blue}{for}}\ \textcolor{BrickRed}{(}\textcolor{ForestGreen}{int}\ te\ \textcolor{BrickRed}{=}\ \textcolor{Purple}{0}\textcolor{BrickRed}{;}\ te\ \textcolor{BrickRed}{\textless{}}\ \textcolor{Purple}{10}\textcolor{BrickRed}{;}\ \textcolor{BrickRed}{++}te\textcolor{BrickRed}{)}\ \textcolor{Red}{\{} \\
\mbox{}\ \ \ \ \ \ \ \ \textbf{\textcolor{Blue}{for}}\ \textcolor{BrickRed}{(}\textcolor{ForestGreen}{int}\ i\ \textcolor{BrickRed}{=}\ \textcolor{Purple}{1}\textcolor{BrickRed}{;}\ i\ \textcolor{BrickRed}{\textless{}=}\ tot\textcolor{BrickRed}{;}\ \textcolor{BrickRed}{++}i\textcolor{BrickRed}{)}\ tree\textcolor{BrickRed}{.}\textbf{\textcolor{Black}{add}}\textcolor{BrickRed}{(}\textbf{\textcolor{Black}{rand}}\textcolor{BrickRed}{());} \\
\mbox{}\ \ \ \ \textcolor{Red}{\}} \\
\mbox{} \\
\mbox{}\ \ \ \ \textbf{\textcolor{Blue}{return}}\ \textcolor{Purple}{0}\textcolor{BrickRed}{;} \\
\mbox{} \\
\mbox{}\textcolor{Red}{\}} \\
\mbox{}

\subsection{TreeSet.java}
\noindent
\mbox{}\textbf{\textcolor{Blue}{public}}\ \textbf{\textcolor{Blue}{class}}\ \textcolor{TealBlue}{MainTreeSet}\ \textcolor{Red}{\{} \\
\mbox{}\ \ \ \ \textbf{\textcolor{Blue}{public}}\ \textbf{\textcolor{Blue}{static}}\ \textcolor{ForestGreen}{void}\ \textbf{\textcolor{Black}{main}}\textcolor{BrickRed}{(}\textcolor{TealBlue}{String}\ args\textcolor{BrickRed}{[])}\ \textcolor{Red}{\{} \\
\mbox{}\ \ \ \ \ \ \ \  \\
\mbox{}\ \ \ \ \ \ \ \ \textcolor{ForestGreen}{int}\ N\ \textcolor{BrickRed}{=}\ Integer\textcolor{BrickRed}{.}\textbf{\textcolor{Black}{parseInt}}\textcolor{BrickRed}{(}args\textcolor{BrickRed}{[}\textcolor{Purple}{0}\textcolor{BrickRed}{]);} \\
\mbox{}\ \ \ \ \ \ \ \ \textcolor{ForestGreen}{int}\ tot\ \textcolor{BrickRed}{=}\ N\ \textcolor{BrickRed}{/}\ \textcolor{Purple}{10}\textcolor{BrickRed}{;} \\
\mbox{} \\
\mbox{}\ \ \ \ \ \ \ \ \textcolor{TealBlue}{Random}\ random\ \textcolor{BrickRed}{=}\ \textbf{\textcolor{Blue}{new}}\ \textbf{\textcolor{Black}{Random}}\textcolor{BrickRed}{();} \\
\mbox{}\ \ \ \ \ \ \ \ \textcolor{TealBlue}{TreeSet\textless{}Integer\textgreater{}}\ tree\ \textcolor{BrickRed}{=}\ \textbf{\textcolor{Blue}{new}}\ TreeSet\textcolor{BrickRed}{\textless{}}Integer\textcolor{BrickRed}{\textgreater{}();} \\
\mbox{}\ \ \ \ \ \ \ \ \textbf{\textcolor{Blue}{for}}\ \textcolor{BrickRed}{(}\textcolor{ForestGreen}{int}\ te\ \textcolor{BrickRed}{=}\ \textcolor{Purple}{0}\textcolor{BrickRed}{;}\ te\ \textcolor{BrickRed}{\textless{}}\ \textcolor{Purple}{10}\textcolor{BrickRed}{;}\ \textcolor{BrickRed}{++}te\textcolor{BrickRed}{)}\ \textcolor{Red}{\{} \\
\mbox{}\ \ \ \ \ \ \ \ \ \ \ \ \textbf{\textcolor{Blue}{for}}\ \textcolor{BrickRed}{(}\textcolor{ForestGreen}{int}\ i\ \textcolor{BrickRed}{=}\ \textcolor{Purple}{1}\textcolor{BrickRed}{;}\ i\ \textcolor{BrickRed}{\textless{}=}\ tot\textcolor{BrickRed}{;}\ \textcolor{BrickRed}{++}i\textcolor{BrickRed}{)}\ tree\textcolor{BrickRed}{.}\textbf{\textcolor{Black}{add}}\textcolor{BrickRed}{(}random\textcolor{BrickRed}{.}\textbf{\textcolor{Black}{nextInt}}\textcolor{BrickRed}{());} \\
\mbox{}\ \ \ \ \ \ \ \ \textcolor{Red}{\}} \\
\mbox{} \\
\mbox{}\ \ \ \ \textcolor{Red}{\}} \\
\mbox{}\textcolor{Red}{\}} \\
\mbox{}


%-----------------------------------------------------------
\end{document}
