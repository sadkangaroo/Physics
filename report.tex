\documentclass[a4paper,14pt]{report}
%
%--------------------   start of the 'preamble'
%
\setlength{\parindent}{2em}
\usepackage[bottom]{footmisc}
\usepackage{graphicx,amssymb,amstext,amsmath}
\usepackage{adobe_fonts}
\usepackage{report_title}
\usepackage[table,usenames,dvipsnames,svgnames]{xcolor}

%
%%    homebrew commands -- to save typing
\setlength{\parskip}{0.7em}
\newcommand\etc{\textsl{etc}}
\newcommand\eg{\textsl{eg.}\ }
\newcommand\etal{\textsl{et al.}}
\newcommand\Quote[1]{\lq\textsl{#1}\rq}
\newcommand\fr[2]{{\textstyle\frac{#1}{#2}}}
\newcommand\miktex{\textsl{MikTeX}}
\newcommand\comp{\textsl{The Companion}}
\newcommand\nss{\textsl{Not so Short}}
%
%---------------------   end of the 'preamble'
%

\renewcommand{\abstractname}{\LARGE{Abstract}}

\begin{document}
%-----------------------------------------------------------
\Letsmaketitle
%-----------------------------------------------------------
\begin{abstract}\centering
\vspace{3mm}
为什么不同乐器听起来的声音不同?为什么人耳能够区分同一频率的不同乐器?为什么人耳能区分同时奏响的不同频率的声音?为什么从声音中\\能听到情感的表达?自然界的声音可以完全用简洁的\\数学表达吗?音乐数字化会遇到哪些问题?
\end{abstract}
%-----------------------------------------------------------
\tableofcontents
%-----------------------------------------------------------
\chapter{动机阐述}
\section{总体概述}
物理学引论(1)的课程中,在讲解振动的知识时涉及了一些声学的内容,以前在大家看来非常神奇的声现象都得到了初步的解释,尤其是许多学乐器或者爱听音乐的同学,在听了泛音这部分内容后,对音乐有了更深刻的理解。
\par
在本次研究中,我们从频谱图出发,对我们感兴趣的一些问题,如有没有客观存在的音高,人耳是如何理解声音的,尤其是如何同时听出不同音色和不同音高,频谱如何体现情感等问题进行研究,并得出一些初步的结论和进一步的猜想。
\section{研究动机}
这次大作业之所以选择这个题目,是因为我们认为作为非物理专业的学生,学习物理的主要目的是学习物理的思想,并对其他方面的学习产生指导作用,另一方面,也通过物理更深刻的了解这个世界,更深入的理解生活。
\par
因此,我们选取了我们认为最贴近生活的主题——音乐,来进行研究。在进行一定量的科学分析之后,我们认为更主要的是深入的思考,是更贴近于对哲学与生活的思考,以及如何解释身边的现象,如何认识科学与感知的关系等。

\chapter{研究过程}
\section{对不同乐器频谱图的分析}


\chapter{成果概述}
\section{总结}

\begin{itemize}
    \item 通过分析五种乐器的频谱图,巩固了对泛音列的学习, 并对泛音列的一些特性进行了研究
    \item 通过观察大提琴的波形图,认识到同一乐器不同音高的音波形也可以有很大差别,音的起始部分变化剧烈十分复杂,延音部分有很强的规律可循
    \item 通过对多个波形图与实际的对比,得出决定客观听觉音高的是波形的最小正周期,并研究了“基音消失”的现象,再次确认了最小正周期才是决定听觉音高的要素
    \item 通过分析不同频率波形的合成图,对人耳辨别同时奏响的不同频率的声音的能力的原理进行了猜想,并对辨别不同乐器的能力也进行了解释。
    \item 通过查阅资料,知道了早期的音乐数字化通过FM实现,后来通过波表实现,并对波表的要求进行了分析
    \item 对音乐与听感的关系进行了思考与解释
\end{itemize}

\section{感谢}
\begin{itemize}
    \item 感谢李晟老师一学期的辛勤教学,让我们受益良多。尤其是关于物理与哲学的思辨给了我们许多指引
    \item 感谢组员们的相互配合,共同完成这次大作业
\end{itemize}

%-----------------------------------------------------------
\end{document}
