\chapter{成果概述}
\section{总结}

\begin{itemize}
    \item 通过分析五种乐器的频谱图,巩固了对泛音列的学习, 并对泛音列的一些特性进行了研究
    \item 通过观察大提琴的波形图,认识到同一乐器不同音高的音波形也可以有很大差别,音的起始部分变化剧烈十分复杂,延音部分有很强的规律可循
    \item 通过对多个波形图与实际的对比,得出决定客观听觉音高的是波形的最小正周期,并研究了“基音消失”的现象,再次确认了最小正周期才是决定听觉音高的要素
    \item 通过分析不同频率波形的合成图,对人耳辨别同时奏响的不同频率的声音的能力的原理进行了猜想,并对辨别不同乐器的能力也进行了解释。
    \item 通过查阅资料,知道了早期的音乐数字化通过FM实现,后来通过波表实现,并对波表的要求进行了分析
    \item 对音乐与听感的关系进行了思考与解释
\end{itemize}

\section{感谢}
\begin{itemize}
    \item 感谢李晟老师一学期的辛勤教学,让我们受益良多。尤其是关于物理与哲学的思辨给了我们许多指引
    \item 感谢组员们的相互配合,共同完成这次大作业
\end{itemize}
