\chapter{动机阐述}
\section{总体概述}
物理学引论(1)的课程中,在讲解振动的知识时涉及了一些声学的内容,以前在大家看来非常神奇的声现象都得到了初步的解释,尤其是许多学乐器或者爱听音乐的同学,在听了泛音这部分内容后,对音乐有了更深刻的理解。
\par
在本次研究中,我们从频谱图出发,对我们感兴趣的一些问题,如有没有客观存在的音高,人耳是如何理解声音的,尤其是如何同时听出不同音色和不同音高,频谱如何体现情感等问题进行研究,并得出一些初步的结论和进一步的猜想。
\section{研究动机}
这次大作业之所以选择这个题目,是因为我们认为作为非物理专业的学生,学习物理的主要目的是学习物理的思想,并对其他方面的学习产生指导作用,另一方面,也通过物理更深刻的了解这个世界,更深入的理解生活。
\par
因此,我们选取了我们认为最贴近生活的主题——音乐,来进行研究。在进行一定量的科学分析之后,我们认为更主要的是深入的思考,是更贴近于对哲学与生活的思考,以及如何解释身边的现象,如何认识科学与感知的关系等。
